\documentclass[12pt]{zjutbook}

\zjutTitle
  {论文中文题目}
  {Title of Thesis}
\zjutAuthor{作者姓名}
\zjutMentor{教师姓名}{教  授}
\zjutMajor{软件工程}
\zjutAcademic{工程硕士} % 工学硕(博)士;专硕:工程硕士
\zjutCultiviate{全日制专业学位硕士} % 全日制专业学位硕士/全日制学术型硕(博)士
\zjutCollege{计算机科学与技术学院}
\zjutDate{2022}{06}    % 夏季06 春季01

\begin{document}
%%%%%%%% 封面 %%%%%%%%
\zjutpreface
%%%%%%%% 中文摘要 %%%%%%%%
\pagenumbering{Roman} % 摘要页码为大写罗马数字

\begin{abstractcn}
  摘要内容,小四号宋体,段前段后 0 磅,1.5 倍间距。500 字左右。每段开头
  空两格,标点符号占一格。中文摘要应表达毕业设计工作的核心内容,简短明了。

  首先,摘要应当要素齐全。即一篇摘要应当包含如下要素:\whiteding{1} 目的—即从
  事该项研究开发的理由与背景或所涉及的主题范围;\whiteding{2} 方法—即所用的原理、
  理论、开发工具,关键技术解决方法等;\whiteding{3} 结果—即研究开发工作的结果、数
  据、效果、性能等;\whiteding{4} 结论—即对结果的分析、评价等。

  其次,摘要应当客观、如实地反映论文的内容。

  第三,采用第三人称写法。由于摘要将直接被检索类二次文献采用,脱离原
  文独立存在,所以摘要一律采用第三人称写法。

  \keywordscn{具体关键词,小四号宋体,段前段后 0 磅,1.5 倍间距;关键词数量为 4—6个,每一关键词之间用逗号分开,最后一个关键词不用标点符号}
\end{abstractcn}


%%%%%%%% 英文摘要 %%%%%%%%
\begin{abstracten}
  The graffiti period is the beginning of children's self-expression. Graffiti is a product of visual experience, body and finger muscle movement. It also reflects the child's physical and mental state. Therefore, Children's graffiti products are a hot issue in entertainment product.

  \keywordsen{Children's graffiti, deep learning, pool function, network transmission and communication, dynamic link library}
\end{abstracten}


%%%%%%%% 目录 %%%%%%%%
\frontmatter
\onehalfspacing
\tableofcontents
\clearpage
\listoffigures
\clearpage
\listoftables


%%%%%%%% 正文 %%%%%%%%
\mainmatter
\doublespacing
\chapter{绪论}
\section{研究背景和意义}
绪论部分主要阐述论文选题的意义,说明研究的必要性、学术价值和实际意义,有具体项目背景的还说明项目来源[1]。
\section{研究目的}
\section{研究方案}
\section{预期目标}

\chapter{正文字体和公式用例}
\section{字体说明}
\begin{enumerate}
  \item 一级标题,中文黑体三号,英文arial三号。居中,1.25倍行距,段前段后各0磅,上面空1行,下面空2行
  \item 二级标题,中文黑体四号,英文Times New Roman四号。两端对齐,1.25倍行距,段前24磅段后12磅
  \item 三级标题,中文黑体小四号,英文Times New Roman小四号。两端对齐,1.25倍行距,段前段后各0磅
  \item 正文,中文宋体小四号,英文Times New Roman小四号。两端对齐,首行缩进2字符,1.25倍行距,段前段后各0磅
\end{enumerate}

\section{公式例子}
\subsection{前向传播算法}
前向传播算法其过程都可以用如下公式表示:
\begin{equation}
  z^{i+1}=\sigma(a^{i+1})=\sigma(x^i\star w^{i+1}+b^{i+1})
\end{equation}

其中,上标代表层数,星号$\star$表示卷积操作,$b$表示偏置项,$\sigma$表示激活函数。
常见的激活函数如ReLu、Sigmoid、Tanh,详细在下一节中介绍。

\subsection{稀疏表示}
给定 $m$ 维特征向量 $X\in R^m$,它代表一段信号或是一副图片。
另外设定由基本组成元素构成的标准化基础矩阵 $D\in R^{m\times p}$,即字典 $D$,
它在信号中是不同频率的波形,在图像中则是构成图像的基本边和角。
$X$ 可以由矩阵 $D$ 中少量的列向量进行线性组合而得到,
其表示系数的矩阵为稀疏矩阵 $\alpha\in R^p$,如\autoref{eq:sparse}所示:

\begin{equation}
  X=D\alpha
  \label{eq:sparse}
\end{equation}

\chapter{图表规范和参考用例}
\section{表格规范和参考用例}
表格采用三线表,首末线1.5磅,第二线0.5磅,文字五号,单倍行距,行高0.7-0.8cm左右

\section{算法性能比较表格}
\section{系统目录规划表格}

\chapter{参考文献和参考用例}

参考文献应按文中引用出现的顺序列全,附于文末。
学位论文中列出的参考文献必须实事求是,论文中引用的必须列出,没引用的一律删去。

参考文献字体用5号,中文用宋体,英文用Times New Roman,行间距选1.4倍。

根据GB 3469 规定,以单字母方式标识各种参考文献类型:

\begin{table}[htp]
  \centering
  \begin{tabular}{cc}
    \toprule
    \textbf{参考文献类型} & \textbf{名称} \\
    \midrule
    专著              & M           \\
    论文集             & C           \\
    析出论文            & A           \\
    报纸文章            & N           \\
    期刊文章            & J           \\
    学位论文            & D           \\
    报告              & R           \\
    标准              & S           \\
    专利              & P           \\
    其他文献            & Z           \\
    \bottomrule
  \end{tabular}
\end{table}

对于数据库,计算机程序及光盘图书等电子文献类型的参考文献,以下列字母为标识:
\begin{table}[htp]
  \centering
  \begin{tabular}{cccc}
    \toprule
    \textbf{参考文献类型} & \textbf{文献类型标识} \\
    \midrule
    数据库(网上)         & DB (DB/OL)      \\
    计算机程序(磁盘)       & CP (CP/DK)      \\
    光盘图书            & M/CD            \\
    \bottomrule
  \end{tabular}
\end{table}

用到5类文献码,分别是[C](会议)、[J] (杂志)、[M](书籍)、[EB/OL](在线刊物)、[D](研究生论文)。注意:1)学术会议的文献标识码为[C],不是[A];2)不要出现太多的D和M,会让专家感觉研究不够深入;3) 一般参考文献50篇以上,要有中文和英文参考文献,而且英文参考文献要在20篇以上,否则专家会觉得对国内外研究现状不了解;4)中文参考文献中尽量不要出现低档次期刊(例如某某大学学报(普通高校的学报通常质量一般)、计算机工程与应用、计算机应用研究等),尽量引用高水平期刊(例如计算机学报、软件学报、计算机研究与发展、电子学报、电子与信息学报等)5)外国人名字按照该写法:Igarashi T;6)近5年的参考文献要占到1/3以上 7)参考文献不同类型写法;
1)会议写法:Tang K K, Song P, Chen X P. Signature of geometric centroids for 3D local shape description and partial shape matching[C], Proceedings of the Asian Conference on Computer Vision. Heidelberg: Springer, 2016, 10115: 311-326
2)杂志写法:Song C Z, Yuille A. Region competition: unifying snakes, region growing, and Bayes/MDL for multiband image segmentation[J]. IEEE Transactions on Pattern Analysis and Machine Intelligence, 1996, 18(9):884-900
3)论文写法:陈国栋. 基于代理模型的多目标优化方法及其在车身设计中的应用[D]. 湖南大学, 2012.
4)书籍写法:中国能源中长期发展战略研究项目组. 中国能源中长期 (2030、2050) 发展战略研究[M]. 科学出版社, 2011.
在线刊物写法:王明亮. 关于中国学术期刊标准化数据库系统工程的进展 [EB/OL]. http://www.cajcd.edu.cn/pub/wml.txt/980810-2.html,  1998-08-16

\section{测试引用}
测试引用:
\cite{zhangkun1994}
\cite{zhukezhen1973}
\cite{dupont1974bone}
\cite{zhengkaiqing1987}
\cite{jiangxizhou1980}
\cite{jianduju1994}
\cite{merkt1995rotational}
\cite{mellinger1996laser}
\cite{bixon1996dynamics}
\cite{mahui1995}
\cite{carlson1981two}
\cite{taylor1983scanning}
\cite{taylor1981study}
\cite{shimizu1983laser}
\cite{atkinson1982experimental}
\cite{kusch1975perturbations}
\cite{guangxi1993} \\
\cite{huosini1989guwu}
\cite{wangfuzhi1865songlun}
\cite{zhaoyaodong1998xinshidai}
\cite{biaozhunhua2002tushu}
\cite{chubanzhuanye2004}
\cite{who1970factors}
\cite{peebles2001probability}
\cite{baishunong1998zhiwu}
\cite{weinstein1974pathogenic}
\cite{hanjiren1985lun}
\cite{dizhi1936dizhi}
\cite{tushuguan1957tushuguanxue}
\cite{aaas1883science}
\cite{fugang2000fengsha}
\cite{xiaoyu2001chubanye}
\cite{oclc2000about}
\cite{scitor2000project}

\chapter{研究生论文专家评审常见问题及修改方案}
论文评审是对研究生毕业前的一道重要程序。
也是决定研究生能否顺利毕业的前提和关键。
每年都有研究生因为论文评审不通过无法正常毕业。
为了帮助研究生顺利通过论文评审。
在这里,我们汇总了评审过程中的一些常见问题以及相应的修改方案,
以帮助研究生顺利通过论文评审。

\section{专家评审要点}
\begin{enumerate}
  \item 选题有重要的理论意义或工程实用价值;
  \item 综述能够对已有研究进行全面总结和分析存在的问题; 解决研究问题的方案设计合理;
  \item 所提出的理论或方法上有创新性或者是技术方面有明显的突破;
  \item 工作量饱满,并且技术路线上有较高的实现难度;
  \item 论文写作条理清晰,分析严谨,图表规范、参考文献引用规范
\end{enumerate}
\section{专家评审典型意见及建议修改方案}
本节对于历年专家反馈回来的典型意见进行汇总并给出建议修改意见,供各位研究生在撰写论文时参考。
\subsection{中文摘要方面的问题}
专家意见:
1)摘要背景知识介绍过多,论文所取得的成果没有体现;
2)建议作者在突出创新点的基础上按目的、方法、结果和结论撰写摘要。

修改方案:摘要通常第一段写背景、意义。
从第二段开始写自己的工作。
需要体现作者本人的工作,未说明所完成工作的应用效果等
(成果例如该算法把识别准确率提高了???;该系统已经在???应用,减少了???。)。
同时需避免的一些常见问题:
书写时,注意要符合中文语法,做到语句通顺、言简意赅、一定要使用书面语言,去口语化,
例如:不要出现“我”或者 “我们”,“这个”用“该”代替,“也就是说”用“即”,
“它的功能”用“其功能”代替等等。
第一次出现缩写的地方给出全称,如:DBS(Database system,数据库系统)。

\chapter{结论与展望}
\section{结  论}
简要回顾所做的工作,包括为什么要做这件研究工作,
采用什么方法,做了哪些事,取得了哪些结果,
是否有实验、仿真或实际应用,效果如何?
对推进本学科发展有什么作用?
要注意不要完全复制摘要,既有类似之处,但也不完全相同。
\section{展  望}
可结合技术发展趋势,分析本文尚存在的问题,
简要说明下一步可如何做以解决这些存在的问题,
同时展望一下该方向的发展前景。

%%%%%% 参考文献 %%%%%%
\backmatter
\bibliography{bib.bib}


%%%%%% 致谢 %%%%%%
\chapter{致  谢}
时光如梭,岁月如歌,接到研究生入学通知书时的场景还历历在目,转瞬又将离开校园,一别经年。

%%%%%% 作者简介 %%%%%%
\chapter{作者简介}
\section{作者简历}
××××年××月出生于××××。

××××年××月——××××年××月,××大学××院(系)××专业学习,攻读××学硕士学位。

\section{攻读硕士学位期间发表的学术论文}
[1]	Jia M M, Yi N N, Bing O, Ding P Q, Wu R.. Maximum spatial-temporal isometric cluster for dynamic surface correspondence, The Visual Computer, 2019, accepted. (SCI源期刊,IF = X.XX)

[2]	Jia M M, Yi N N, Bing O, Ding P Q. Article Title. Journal Title, 2015, 118(1/2): 389–398. (SCI收录,IDS号为XXXXX,IF= X.XX)

[3]	贾某某, 易某某, 邴某某, 等. 多视图卷积网络加权优化. 计算机辅助设计与图形学学报, 2018, xx(x): xx–xx. (EI收录号:2014XXXXXXXXX)

\section{参与的科研项目及获奖情况}
[1]	易某某, 贾某某. ××××××××××, 国家自然科学基金项目. 编号: ××××.

[2]	易某某, 贾某某. ××××××××××××××××××. ××省科学技术一等奖, 2014.

\section{发明专利}
[1]	贾某某. ××××××××××××. 中国, 2013 1 0513271.2 [P]. 2015-04-26.

%%%%%% 附录 %%%%%%
\chapter{学位论文数据集}
\end{document}

\pagenumbering{Roman}

\begin{abstractcn}
摘要内容,小四号宋体,段前段后 0 磅,1.5 倍间距。500 字左右。每段开头
空两格,标点符号占一格。中文摘要应表达毕业设计工作的核心内容,简短明了。

首先,摘要应当要素齐全。即一篇摘要应当包含如下要素:\whiteding{1} 目的—即从
事该项研究开发的理由与背景或所涉及的主题范围;\whiteding{2} 方法—即所用的原理、
理论、开发工具,关键技术解决方法等;\whiteding{3} 结果—即研究开发工作的结果、数
据、效果、性能等;\whiteding{4} 结论—即对结果的分析、评价等。

其次,摘要应当客观、如实地反映论文的内容。

第三,采用第三人称写法。由于摘要将直接被检索类二次文献采用,脱离原
文独立存在,所以摘要一律采用第三人称写法。

\keywordscn{具体关键词,小四号宋体,段前段后 0 磅,1.5 倍间距;关键词数量为 4—6个,每一关键词之间用逗号分开,最后一个关键词不用标点符号}
\end{abstractcn}

\begin{abstracten}
Middle check system in teaching reform and bulid project is applied to realize
project middle examine on-line. The use of the system will change the traditional
method of project examine type, control the implementation process of project and
promise the quality of project……

(小四, Times New Roman, 段前段后 0 磅,1.5 倍行间距)

每段开头留 4 个字符空格,英文摘要的内容应与中文摘要基本相对应,

\keywordsen{全部小写,每一关键词之间逗号分开,最后一个关键词后不打标点符号;字体小四, Times New Roman, 段前段后 0 磅,1.5 倍行间距)}
\end{abstracten}